Content of Video:
\begin{itemize}
    \item The general theme of our video will the unique goal of Scratch compared to other programming languages, how Scratch looks, and some of the unique features 
    \item Scratch is made for kids, and lots of aspects of the interface help them understand how code works
    \item Scratch is intentionally simplified and limited in its functions so that kids can understand all of their tools
    \item The main coding environment is scratch.mit.edu which serves not only as an IDE, but also as a community for kids to learn and be creative
    \item The three data types in Scratch are numbers, strings, and Booleans
    \item Functions have extensive polymorphism which allows code to work as an inexperienced programmer would probably intend (also why variables have type inference)
    \item Errors are handled as invisibly to the user as possible, and usually won't stop the execution of the program
    \item Scratch has extensive support for visuals and audio
    \item We will demonstrate the final big program, in our case we created a simple game to show off the capabilities of the Scratch language.  
\end{itemize}

Logistics of Video:
\begin{itemize}
    \item The video will be produced using screen capture and PowerPoint slides with our voice over.
    \item Our video will have audio recorded over slides that explain what Scratch is and have simple code fragments which encapsulate the unique features of Scratch
    \item We will have a code demonstration of our project, which will explore features in a real example as we explain what what's happening
    \item We will demo our program in Scratch's online interface
\end{itemize}